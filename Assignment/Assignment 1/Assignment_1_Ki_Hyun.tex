%=======================02-713 LaTeX template, following the 15-210 template==================
%
%
%
%
%    1. Update the information in section "A," put your name and the name of 
%       the class and the number of the problem set
%    2. Write your answers in section "B" below. Precede answers for all 
%       parts of a question with the command "\question{n}{desc}" where n is
%       the question number and "desc" is a short, one-line description of 
%       the problem. There is no need to restate the problem.
%    3. If a question has multiple parts, precede the answer to part x with the
%       command "\part{x}".
%

\documentclass[11pt]{article}
\usepackage[mathscr]{euscript}


\newcommand\question[2]{\vspace{.25in}\hrule\textbf{#1: #2}\vspace{.5em}\hrule\vspace{.10in}}
\renewcommand\part[1]{\vspace{.10in}\textbf{(#1)}}


% plots
\usepackage{graphicx}
\usepackage{multirow}
\usepackage{tabularx}
\graphicspath{ {./images/} }


\usepackage{amsmath, amssymb, amsthm} %AMS packages
\usepackage{mathtools, graphicx, enumitem} %generally recommended
\usepackage{mathrsfs} %symbols I like
\usepackage{hyperref} %personal preference
\usepackage{microtype, } %recommended by stackexchange, never tried them myself
\usepackage{nag, todonotes} %workflow stuff, doesn't affect final document

% Layout
\usepackage{fancyhdr}
\usepackage[margin=1in]{geometry}
\lhead{\NAME}
\chead{\ClassNumber , Assignment \ANUM}
\rhead{Due: \duedate}
\setlength{\parindent}{0pt}
\setlength{\parskip}{5pt plus 1pt}
\setlength{\headheight}{13.6pt}
\pagestyle{fancyplain}


%-------------------------------------------<commands>--------------------------------------------------------
%%%%%%%%%%%%%%%%%%%%%%%%%%%%%%%%%%%%%%%%%											Letter Symbols
\newcommand{\R}{\mathbb{R}}
\newcommand{\N}{\mathbb{N}}
\newcommand{\Z}{\mathbb{Z}}
\newcommand{\Q}{\mathbb{Q}}
\newcommand{\C}{\mathbb{C}}
\newcommand{\F}{\mathcal{F}}
\renewcommand{\H}{\mathcal{H}} %overwrites long-umlaut diacritic
\newcommand{\eps}{\varepsilon}
\newcommand{\Exp}{\mathbb{E}}
\newcommand{\Info}{\mathcal{F}}
\renewcommand{\P}{\mathbb{P}}
%%%%%%%%%%%%%%%%%%%%%%%%%%%%%%%%%%%%%%%%%											Brackets
\newcommand{\paren}[1]{\left( #1 \right)}
\newcommand{\bracket}[1]{\left[ #1 \right]}
\newcommand{\chevron}[1]{\langle #1 \rangle}
\newcommand{\norm}[2][ ]{\left\lVert #2 \right\rVert_{#1}}
\newcommand{\abs}[1]{\left\lvert #1 \right\rvert}
\newcommand{\floor}[1]{\left\lfloor #1 \right\rfloor}
\newcommand{\ceil}[1]{\left\lceil #1 \right\rceil}
%%%%%%%%%%%%%%%%%%%%%%%%%%%%%%%%%%%%%%%%%											Operators
\DeclareMathOperator{\supp}{supp}
\DeclareMathOperator{\trace}{tr}
\DeclareMathOperator{\lspan}{span}
\DeclareMathOperator{\conv}{conv} % stands for conv, as in convex hull
\DeclareMathOperator{\Int}{int} % stands for int, as in interior of a set
\DeclareMathOperator{\cl}{cl}
\DeclareMathOperator{\sgn}{sgn}
\newcommand{\indep}{\perp \!\!\! \perp}
\newcommand{\NormCDF}[1]{\Phi \left[ #1 \right]}
%%%%%%%%%%%%%%%%%%%%%%%%%%%%%%%%%%%%%%%%%%											Aarows
\newcommand{\into}{\hookrightarrow}
\newcommand{\onto}{\twoheadrightarrow}
\newcommand{\weakly}{\rightharpoonup}
\newcommand{\isom}{\cong}
\newcommand{\restr}[1]{{\upharpoonright}_{#1}}
%\newcommand{\rest}{\big|}
%%%%%%%%%%%%%%%%%%%%%%%%%%%%%%%%%%%%%%%%%%											Common Abbreviations
\renewcommand{\th}{^\mathrm{th}} %overwrites thorn (old english letter)
\newcommand{\n}{^{-1}}
\newcommand{\half}{\frac{1}{2}}
%%%%%%%%%%%%%%%%%%%%%%%%%%%%%%%%%%%%%%%%%%											Differential Operators
\newcommand{\del}{\partial}
\newcommand{\grad}{\nabla}
\newcommand{\Laplace}{\Delta}
\renewcommand{\div}{\operatorname{div}} %overwrites division symbol
\newcommand{\dd}{\mathrm{d}}
\newcommand{\intd}{\,\dd}
\newcommand{\ddt}{\frac{\dd}{\dd t}}
\newcommand{\deriv}[2]{\frac{\dd #1}{\dd #2}} %can leave top blank, e.g. \deriv{}{x}
\newcommand{\pderiv}[2]{\frac{\partial #1}{\partial #2}} %ditto
%%%%%%%%%%%%%%%%%%%%%%%%%%%%%%%%%%%%%%%%%%											Favorite Functions and Space Names
\newcommand{\test}{\mathcal{D}}
\newcommand{\Ctest}{C_c^\infty}
\newcommand{\BMO}{\operatorname{BMO}}
\newcommand{\indic}[1]{\chi_{\{#1\}}}
\newcommand{\Leb}[2][\R^n]{L^{#2}(#1)}
%-------------------------------------------</commands>--------------------------------------------------------


\begin{document}\raggedright


%Section A==============Change the values below to match your information==================
\newcommand\NAME{Ki Hyun}  % your name
\newcommand\ClassNumber{FINM 36702}
\newcommand\ClassName{Portfolio Credit Risk: Modeling and Estimation}    
\newcommand\ANUM{1}              % the homework number
\newcommand\duedate{18:00 (CT) March 30th 2023}	% due date
%Section B==============Put your answers to the questions below here=======================

\title{Assignment \ANUM}
\author{\NAME \\ 
\ClassNumber \text{:} \ClassName}
\date{Due: \duedate}

\maketitle

%\question{1}{Description of problem}
%\part{a}

\question{1}{Correlation and Default Correlation}

\begin{center}
\begin{tabular}{||c | c | c||} 
 \hline
 $\rho_{1,2}$ & $\rho_{1,3}$ & $\rho_{2,3}$ \\
 \hline
 0.60 & 0.43 & -6.5 $\times 10^{-9}$ \\ 
 \hline
\end{tabular}
\end{center}

\begin{center}
\begin{tabular}{||c | c | c||} 
 \hline
 $Corr[D_1, D_2]$ & $Corr[D_1, D_3]$ & $Corr[D_2, D_3]$ \\
 \hline
 0.33 & 0.22 & 0.00 \\ 
 \hline
\end{tabular}
\end{center}

\question{2}{PDJ with Gauss Copula Assumption}

\part{i: PDJ}

\begin{center}
\begin{tabular}{||c | c | c||} 
 \hline
 $PDJ_{1,2}$ & $PDJ_{1,3}$ & $PDJ_{2,3}$ \\
 \hline
 0.027 & 0.032 & 0.039 \\ 
 \hline
\end{tabular}
\end{center}

\part{ii: Range}

We know that 

$$
\P\{D_i \cap D_j\} = PDJ_{i, j}
$$

Moreover, due to basic set theory, $\forall(i, j)$:

$$
\P\{D_1 \cap D_2 \cap D_3\} \leq \P\{D_i \cap D_j\}
$$

$$
\therefore
\P\{D_1 \cap D_2 \cap D_3\} \leq \min_{(i, j)} \P\{D_i \cap D_j\}
= PDJ_{1, 2}
\approx 0.027
$$

Therefore, the probability that all three firms default ranges from
$0$ to $0.027$

\part{iii: All Default}

Under Gauss copula, the probability that all three default is
$\approx 0.016$

\question{3}{Firm "A" and Firm "B"}

\begin{center}
	\begin{tabular}{c c|c|c|c}
  	\cline{2-5}
  	\multirow{3}{*}{\rotatebox[origin=c]{90}{\textbf{Firm "B"}}}
  	& \textbf{A} & 0.02 & 0.16 & 0.32 \\
  	\cline{2-5}
  	& \textbf{B} & 0.05 & 0.19 & 0.16 \\
  	\cline{2-5}
  	& \textbf{D} & 0.03 & 0.05 & 0.02 \\
  	\cline{2-5}
  	& \textbf{} & \textbf{D} & \textbf{B} & \textbf{A} \\
  	\cline{2-5} 
  	\multicolumn{1}{c}{} & \multicolumn{4}{c}{\textbf{Firm "A"}}
	\end{tabular}
\end{center}

\question{4}{Consistency with Gauss copula}

The given situation yields a correlation matrix that is 
approximately:

$$
\left[
\begin{matrix}
1 & 0.31 & 0.24 & 0.18 \\
0.31 & 1 & 0.10 & 0.044 \\
0.24 & 0.10 & 1 & -0.036 \\
0.18 & 0.044 & -0.036 & 1 \\
\end{matrix}
\right]
$$

The eigen-vectors and eigen-values decomposition from python
tells us below.

First, for the eigen-values:

\begin{center}
\begin{tabular}{||c | c | c | c||} 
 \hline
 $\lambda_1$ & $\lambda_2$ & $\lambda_3$ & $\lambda_4$ \\
 \hline
 1.5 & 0.61 & 1.0 & 0.87 \\ 
 \hline
\end{tabular}
\end{center}

The corresponding eigen-vectors are:

$$
v_1 = \left( 
\begin{matrix}
-0.67 \\
-0.74 \\
0.060 \\
0.031
\end{matrix}
\right), \
v_2 = \left( 
\begin{matrix}
-0.55 \\
0.46 \\
-0.047 \\
-0.70
\end{matrix}
\right), \
v_3 = \left( 
\begin{matrix}
-0.42 \\
0.36 \\
-0.56 \\
0.61
\end{matrix}
\right), \
v_4 = \left( 
\begin{matrix}
-0.27 \\
0.33 \\
0.82 \\
0.38
\end{matrix}
\right)
$$

The eigen decomposition tells us that the correlation matrix
has 4 ranks and thus non-singular.

Moreover, the eigen-values are non-negative.

Therefore, we may rule that the correlation matrix is positive
semi-definite.

Together, the connection between the defaults of the four firms
is consistent with a Gauss copula.

\end{document}

$$
\begin{aligned}
\end{aligned}
$$